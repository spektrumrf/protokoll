\documentclass{protokollspektrum}

% \datum{Dag}{Månad}{År}
\datum{14}{9}{2016}

%% Om man vill ändra var mötet hölls kan man ta bort kommentaren och editera argumentet
% \plats{Kafferummet}

% Lista på alla styrelsemedlemmar som var närvarande vid mötet
\styrelse{
	Förnamn1 Efternamn1, % Ordförande
	Förnamn2 Efternamn2, % Viceordförande
	Förnamn3 Efternamn3, % Sekreterare
	Förnamn4 Efternamn4, % Skattmästare
	Förnamn5 Efternamn5, % Programchef
	Förnamn6 Efternamn6, % Studiesekreterare
	Förnamn7 Efternamn7, % Extra styrelsemedlem
	Förnamn8 Efternamn8, % Extra styrelsemedlem
}

%% Om man vill ändra på "övriga närvarande" kan man ta bort kommentaren och editera argumentet
% \medlemmar{Alla andra}


% Se filen "test_protokoll.tex" för mera information
\begin{document}

\punkt{Mötets öppnande}{
	Mötet öppnades klockan 18:06.
}

\punkt{Konstaterande av mötets stadgeenlighet och beslutförhet}{
	Mötet konstaterades beslutfört och stadgeenligt sammankallat.
}

\punkt{Uppläsning och godkännande av föredragningslistan}{
	Föredragningslistan lästes upp och godkändes.
}

\punkt{TBA}{}

\punkt{TBA}{}

\punkt{Meddelanden}{}

\punkt{Övriga ärenden}{}

\punkt{Mötets avslutande}{
	Mötet avslutades klockan 18:25.
}

\end{document}