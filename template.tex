%% För att skriva ett styrelsemötes protokoll ska man ha med det extra argumentet "styrelse" i class definitionen.
%\documentclass[styrelse]{protokollspektrum}
% För att skriva ett måmö protokoll behövs bara class definitionen ensam
\documentclass{protokollspektrum}


% Datumet för mötet. Används i båder "headern" på alla sidor och i rubriken.
% {Dag}{Månad}{År}
%
% Exempel:
%	\datum{13}{2}{2014} 	% 13:e februari 2014
\datum{14}{9}{2016}
\extradatum{28}{8}{2016}


% List på alla styrelsemedlemmar som var närvarande vid mötet
% Man kan "kommentera" bort de som inte är närvarande, med procent-tecknet
\styrelse{
	Förnamn1 Efternamn1, % Ordförande
	Förnamn2 Efternamn2, % Viceordförande
	Förnamn3 Efternamn3, % Sekreterare
	% Förnamn4 Efternamn4, % Skattmästare
	Förnamn5 Efternamn5, % Programchef
	% Förnamn6 Efternamn6, % Studiesekreterare
	Förnamn7 Efternamn7, % Extra styrelsemedlem
	Förnamn8 Efternamn8, % Extra styrelsemedlem
}

%% Om man vill ändra var mötet hölls kan man ta bort kommentaren och editera argumentet
% \plats{Kafferummet}

%% Om man vill ändra på "övriga närvarande" kan man ta bort kommentaren och editera argumentet
\medlemmar{Se bilaga 1.}

% Bilagorna kommer i den här ordningen på i slutet av dokumentet.
\bilaga{Närvarande vid mötet}
\bilaga{En annan mindre viktig bilaga}

% Signaturerna kommer i den här ordningen allra sist i dokumentet. Om det är udda antal hamnar den sista ensam på en rad, annars printas de i par.
\signatur{Ordförande, Förnamn1 Efternamn1}
\signatur{Sekreterare, Förnamn2 Efternamn2}
\signatur{Ordförande, Förnamn3 Efternamn3}
\signatur{Sekreterare, Förnamn4 Efternamn4}
\signatur{Sekreterare, Förnamn5 Efternamn5}

%===========================================================================================
% För att kunna använda den här protokollbottnen måste "protokollspektrum.cls" filen finnas
% i samma mapp som ".tex" filen som editeras.
%
%
% Protokollet består av "punkter", punkterna kan sättas in med två olika kommandon:
%
% \punkt{<title>}{<innehåll>}
%
% Enklaste sättet att sätta in en punkt är att använda det här kommandot. Men om man vill
% sätta in några extra radbyten eller av någon anledning behöver använda mera LaTeX-kod
% för att fylla i innehållet, så kan man använda det andra sättet:
%
% \begin{punkten}{<titel>}
% <innehåll>
% \end{punkten}
%
% Programmet sköter numreringen av punkterna själv, så enda användaren måste göra är att
% sätta dem i rätt ordning.
%
%
% LaTeX bryr sig inte om de flesta tabbar/radbyten, utan man måste specifikt säga åt det
% att använda ett blankt utrymme.
%
% Man kan sätta hur många tommar rader som helst mellan punkterna, men innuti \punkt-
% kommandot är ett av undantagen för tomma rader.
%
% När man skapar PDF:en brukar LaTeX oftast villa göra de temporära hjälpfilerna flera
% gånger, för att alla referenser ska fungera. I det här fallet gäller det främst
% sidnumrerningen.
%
%
% Filen "template.tex" har grundstrukturen för ett protokoll.
%===========================================================================================


\begin{document}

\anteckningstart{En anteckning före första punkten}

% Här är några alternativa sätt att skriva punkter med \punkt-kommandot

\punkt{Mötets öppnande}{Mötet öppnades klockan 18:06.}

\punkt{Mötets öppnande}{
Mötet öppnades klockan 18:06.
}

\punkt{Mötets öppnande}{
	Mötet öppnades klockan 18:06.	% LaTeX bryr sig inte om tabbar i början av rader
}

\punkt{Mötets öppnande}{
	Mötet
	öppnades
	klockan
	18:06.	% LaTeX bryr sig inte heller om radbyten
}

% För att skriva ett radbyte, måste man säga "ny rad"
\punkt{Mötets öppnande}{
	Mötet \\			% Två olika sätt att skriva "ny rad"
	öppnades \newline	%
	klockan
	18:06.	% LaTeX bryr sig inte heller om radbyten
}

%%%% Det här funkar inte!!! Kan inte använda tomma rader i det här kommandot
%%%\punkt{Mötets öppnande}{
%%%
%%%Mötet öppnades klockan 18:06.
%%%}

\punkt{Konstaterande av mötets stadgeenlighet och beslutförhet}{Mötet konstaterades beslutfört och stadgeenligt sammankallat.}

\anteckning{En anteckning mellan punkterna}

\punkt{Uppläsning och godkännande av föredragningslistan}{Föredragningslistan lästes upp och godkändes.}

\punkt{Höstens månadsmöten}{Höstens månadsmöten ordnas 14.9, 12.10, 16.11 och 14.12.}

\punkt{Höstens program}{
	Höstens program är 21.9 Gulisintagning, 24-25.9 Kräftis, 28.9 Gulisäventyret, 8.10 Oktoberfest,
	13-16.10 Sverigeexkursion, 25.10 Campussitz, 29-31.10 Gumtäkts Föreningars Gemensamma
	Kryssningar (GFGK), 11.11 Sits 2, 18.11 Deltagande i publiken av Uutisvuoto, 25-27.11 Bang!-
	turnering, 6.12 Hemligt program och 9.12 Julfest tillsammans med Svenska Naturvetarklubben
	(SvNK).
}

\begin{punkten}{Meddelanden}
Gånget program är

14.5 Vårfest

11.6 Sommarträff 1

18-19.6 Jukolakavlen

30.7 Sommarträff 2

19-27.8 Exkursion till Budapest

28.8 Gulispiknik

2.9 Gulisgrillnng

9.9 Gulissitz

12.9 Kung fu panda-filmkväll.

% Man kan ha tomma rader i det här kommandot (LaTeX-språk: environment, \begin{} + \end{})
\end{punkten}

\punkt{Övrigt}{Helsingfors Universitets Studentkårs (HUS) delegationsval ordnas 26-28.10 och 31.10-2.11. Om
någon är intresserad av att ställa upp, kontakta Förnamn Efternamn senast under nästa vecka.
Kommande program är 21.9 Gulisintagning, 24-25.9 Kräftis, 28.9 Gulisäventyret, 8.10
Oktoberfest och 12.10 Månadsmöte.}

\punkt{Mötets avslutande}{Mötet avslutades klockan 18:25.}


\anteckningslut{En anteckning efter sista punkten}

\end{document}